%%--------------------------------------------------
\section{Toward Smart Society}
%% - - - - - - - - - - - - - - - - - - - - - - - - -

In the roadmap discussion, we count the number of scenarios naively
based on the actual numbers used in our works in this project.
We suppose that we simply apply exhaustive search on these scenarios.
Of course, 
we can apply several methods based on design of experiments
or other optimization/learning methods to reduce the number of scenarios
we should run.
OACIS and CARAVAN also provide facilities to
realize such intelligent and effective functions.

We also need to investigate the cost of thinking part of each agent.
In the evaluation above, we assume that the intelligence of each
agent will not change, so that the complexity of the thinking
in each agent is constant.
But, for further simulation researches, we need to introduce
more sophisticated and complex thinking engine to
realize more intelligent and adaptive behaviors like human.
This is still open issues. 

The multiagent social simulation
is evolving research domain to realize smart societies by IT and AI,
and is still under-establishing phase.
However, requests from application fields become stronger and wider.
So, it is important to determine a measure to know achievements will
be important.  
The roadmaps shown in this article will beocme a testbed to provide
such measures.
Also, the CASSIA framework shown in this chapter
will provide powerful tool to push forward the research on this
roadmaps.


